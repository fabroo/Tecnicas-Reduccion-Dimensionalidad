\documentclass[12pt]{article}%

\usepackage{amsmath,amsthm, amsfonts, amscd , amssymb}

\usepackage{hyperref}

\usepackage{float}
\usepackage{listings}

\usepackage{graphicx}
\usepackage[usenames, dvipsnames]{color}
\usepackage[latin1]{inputenc}
\usepackage[spanish]{babel}%
\usepackage{enumerate}% http://ctan.org/pkg/enumitem
\usepackage{marvosym} %\Biohazard \Radioactivity \Keyboard \Stopsign
\usepackage{manfnt}  %\dbend, \lhdbend,and \reversedvideodbend.     \textdbend, \textlhdbend,and \usepackage{phaistos} %\PHpedestrian
\usepackage{enumitem}

\setcounter{MaxMatrixCols}{30}

%\setlength{\textheight}{23cm} \setlength{\textwidth}{17.5cm}
%\setlength{\topmargin}{-1cm} \setlength{\oddsidemargin}{0cm}

\usepackage{geometry}\geometry{top=2cm,bottom=2cm,left=2.5cm,right=2.5cm}
\hyphenation{}

\newcommand{\convprob}{ \buildrel{p}\over\longrightarrow}

\def \sp {\textquestiondown}
\def \RR {\mathbb{R}}
\def \EE {\mathbb{E}}
\def \VV {\mathbb{V}}



\newcommand{\bx}{\mathbf{x}}
\newcommand{\bX}{\mathbf{X}}
\newcommand\bmu {\mbox{\boldmath $\mu$}}
\newcommand{\cov}{\mathbb{C}\mbox{ov}}
\newcommand\bSi {\mbox{\boldmath $\Sigma$}}
\newcommand\bS {\mathbf S}
\newcommand{\bw}{\mathbf{w}}
\newcommand{\bA}{\mathbf{A}}
\newcommand{\bb}{\mathbf{b}}


\begin{document}
	\noindent
	\textbf{Herramientas de Visualizaci�n- Cuarto bimestre 2025}\\[0.3cm]\textbf{Trabajo Pr�ctico Final }
	\vspace{-0.3cm}
	\newline
	\rule{18cm}{0.2mm}\\
	
	\medskip
	\textbf{Se debe entregar un informe .Rmd en formato pdf con la resoluci�n y resultados del ejercicio, incluyendo todos los gr�ficos que crean pertinentes  y el archivo .Rmd donde se realizaron los c�lculos y se program� la implementaci�n del an�lisis pedido. El trabajo se puede realizar en grupos de 2 integrantes.}
	
	\section{Te�rico}
	 Consideremos un vector aleatorio $(\bx,Y)$, donde $\bx  \in \mathcal X \subset \mathbb R^p$ es el vector de covariables e  $Y$ es la clase, la cual toma valores en $\mathcal Y=\{0, 1\}$. Un clasificador $g$ es una funci\'on $g: \mathcal X \to \mathcal Y$. Cuando observamos un nuevo $\bx$ predecimos la clase  como $g(\bx)$.
	
	\medskip
	Consideramos el \textbf{Error de Clasificaci\'on Medio}  del clasificador $g$ definido como
	$$L(g)=\mathbb P(g(\bx)\not =Y)\, .$$
	
	\begin{enumerate} [label=(\alph*)]
		
		\item Probar que dada la naturaleza binaria de $Y$, el \textbf{Error de Clasificaci\'on Medio}  coincide con el Error Cuadr�tico Medio habitual, es decir
		$$L(g)=\mathbb E((Y - g(\bx))^2)\, .$$
		
		\item Supongamos que las distribuciones condicionales son normales multivariadas, es decir,  $\bx |Y =1 \sim \mathcal{N}(\bmu_1,\bSi_1)$ y   $\bx |Y =0 \sim \mathcal{N}(\bmu_0,\bSi_0)$.  Probar que la regla \'optima resulta
		
		\begin{equation*}
			g^{op}(\bx)=\left\{
			\begin{array}{ll}
				1 & \hbox{si}\; r_1(\bx) \le   r_0(\bx) + 2 \log \frac{\pi_1}{\pi_0} + \log \frac{\left| \bSi_0\right| }{\left| \bSi_1\right| }\\
				0 &\hbox{en}\;  \hbox{c. c.}
			\end{array}
			\right. \,,
		\end{equation*}
		donde para $i=0,1$ se tiene que $\pi_i=P\left(Y=i \right) $, $r_i(\bx)=\left( 
		{\bx-\bmu_i }\right)^{t}\bSi_i ^{-1}\left( {\bx-\bmu_i }\right)$ y $\left| \bSi_i\right|$ es la notaci�n para indicar el determinante  de la matriz $\bSi_i$,  para $i=0,1$.		

		\item Probar que si $\bSi_0=\bSi_1=\bSi$, entonces la regla del �tem anterior resulta
\begin{equation*}
	g^{op}(\bx)=\left\{
	\begin{array}{ll}
		1 & \hbox{si}\; 
			D_1(\bx)  - 2 \log (\pi_1) \leq 	D_0(\bx)  - 2 \log (\pi_0)\\
		0 &\hbox{en}\;  \hbox{c. c.}
	\end{array}
	\right. 
\end{equation*}
donde $	D_i(\bx)= (\bx-\bmu_i)^t \bSi^{-1}(\bx-\bmu_i) $.

		\item Comprobar que si $\pi_0=\pi_1$, entonces la regla del �tem anterior resulta
\begin{equation*}
	g^{op}(\bx)=\left\{
	\begin{array}{ll}
		1 & \hbox{si}\; 
		D_1(\bx) \leq 	D_0(\bx) \\
		0 &\hbox{en}\;  \hbox{c. c.}
	\end{array}
	\right. 
\end{equation*}
es decir:  se clasifica a $\bx$ en la poblaci�n cuya media est� m�s cerca en distancia de Mahalanobis. Si $\bSi=\sigma^2 \mathbb I_p$ (siendo $\mathbb I_p$ la identidad de $p \times p$), �a qu� equivaldr�a esta regla? 


	\item Mostrar que si $\bSi_0$ y $\bSi_1$ no coinciden, entonces la regla �ptima puede escribirse como 		
		\begin{equation*}
			g^{op}(\bx)=\left\{
			\begin{array}{ll}
				1 & \hbox{si}\; 
				\bx^t \bA \bx + \bb^t \bx +c \le 0\\
				0 &\hbox{en}\;  \hbox{c. c.}
			\end{array}
			\right. 
		\end{equation*}
y hallar las expresiones de $\bA$, $\bb$ y $c$.		

	\end{enumerate}
	
	


\end{document}